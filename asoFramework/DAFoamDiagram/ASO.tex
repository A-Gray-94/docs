% MDF architecture diagram produced by the TikZ package
\documentclass{article}
\usepackage{geometry}
\usepackage{amsfonts}
\usepackage{amsmath}
\usepackage{amssymb}

\usepackage{tikz}

% Define the set of tikz packages to be included in the architecture diagram document

\usetikzlibrary{arrows,chains,positioning,scopes,shapes.geometric,shapes.misc,shadows} 

% Set the border around all of the architecture diagrams to be tight to the diagrams themselves
% (i.e. no longer need to tinker with page size parameters)

\usepackage[active,tightpage]{preview}
\PreviewEnvironment{tikzpicture}
\setlength{\PreviewBorder}{5pt}


\begin{document}

% Styles definition outside any picture
% Define all the styles used to produce XDSMs for MDO

% Component types
\tikzstyle{Optimization} = [rounded rectangle,draw,fill=blue!20,inner sep=6pt,minimum height=1cm,text badly centered]
\tikzstyle{LP_Optimization} = [rectangle,draw,fill=blue!20,inner sep=6pt,minimum height=1cm,text badly centered]
\tikzstyle{Analysis} = [rectangle,draw,fill=green!20,inner sep=6pt,minimum height=1cm,text badly centered]
\tikzstyle{Function} = [rectangle,draw,fill=purple!20,inner sep=6pt,minimum height=1cm,text badly centered]
\tikzstyle{MDA} = [rounded rectangle,draw,fill=orange!20,inner sep=6pt,minimum height=1cm,text badly centered]
\tikzstyle{Metamodel} = [rectangle,draw,fill=yellow!20,inner sep=6pt,minimum height=1cm,text badly centered]
\tikzstyle{DOE} = [rounded rectangle,draw,fill=yellow!20,inner sep=6pt,minimum height=1cm,text badly centered]
%\tikzstyle{OptFunction} = [rectangle,draw,fill=red!20,inner sep=6pt,minimum height=1cm,text badly centered]

%% A simple command to give the repeated structure look for components and data
\tikzstyle{stack} = [double copy shadow]

%% A simple command to fade components and data, e.g. demonstrating a sequence of steps in an animation
\tikzstyle{faded} = [draw=black!50,fill=white,text opacity=0.5]

%% Simple fading commands for the lines
\tikzstyle{fadeddata} = [color=black!20]
\tikzstyle{fadedprocess} = [color=black!50]

% **OLD** Component types for repeated structures (i.e. for parallel structures)
%\tikzstyle{Optimization_i} = [double copy shadow, Optimization]
%\tikzstyle{LP_Optimization_i} = [double copy shadow, LP_Optimization]
%\tikzstyle{Analysis_i} = [double copy shadow, Analysis]
%\tikzstyle{Function_i} = [double copy shadow, Function]
%\tikzstyle{MDA_i} = [double copy shadow, MDA]
%\tikzstyle{Metamodel_i} = [double copy shadow, Metamodel]
%\tikzstyle{DOE_i} = [double copy shadow, DOE]

% **OLD** Faded component types for, e.g. demonstrations of each step. We use these style definitions to "gray out" large parts of the diagram.
%\tikzstyle{Optimization_fade} = [Optimization,fill=blue!10,draw=black!30,text opacity=0.3]
%\tikzstyle{Analysis_fade} = [Analysis,fill=green!10,draw=black!30,text opacity=0.3]
%\tikzstyle{Function_fade} = [Function,fill=purple!10,draw=black!30,text opacity=0.3]
%\tikzstyle{MDA_fade} = [MDA,fill=orange!10,draw=black!30,text opacity=0.3]
%\tikzstyle{Metamodel_fade} = [Metamodel,fill=yellow!10,draw=black!30,text opacity=0.3]
%\tikzstyle{DOE_fade} = [DOE,fill=yellow!10,draw=black!30,text opacity=0.3]
%
%\tikzstyle{Optimization_i_fade} = [Optimization_i,fill=blue!10,draw=black!30,text opacity=0.3]
%\tikzstyle{Analysis_i_fade} = [Analysis_i,fill=green!10,draw=black!30,text opacity=0.3]
%\tikzstyle{Function_i_fade} = [Function_i,fill=purple!10,draw=black!30,text opacity=0.3]
%\tikzstyle{MDA_i_fade} = [MDA_i,fill=orange!10,draw=black!30,text opacity=0.3]
%\tikzstyle{Metamodel_i_fade} = [Metamodel_i,fill=yellow!10,draw=black!30,text opacity=0.3]
%\tikzstyle{DOE_i_fade} = [DOE_i,fill=yellow!10,draw=black!30,text opacity=0.3]

% Data types
\tikzstyle{DataInter} = [trapezium,trapezium left angle=75,trapezium right angle=105,draw,fill=black!10]
\tikzstyle{DataIO} = [trapezium,trapezium left angle=75,trapezium right angle=105,draw,fill=white]

% **OLD** Data types for repeated structures
%\tikzstyle{DataInter_i} = [double copy shadow, DataInter]
%\tikzstyle{DataIO_i} = [double copy shadow, DataIO]

% **OLD** Faded data types
%\tikzstyle{DataInter_fade} = [DataInter,draw=black!30,fill=white,text opacity=0.3]
%\tikzstyle{DataIO_fade} = [DataIO_i,draw=black!30,fill=white,text opacity=0.3]
%
%\tikzstyle{DataInter_i_fade} = [DataInter_i,draw=black!30,fill=white,text opacity=0.3]
%\tikzstyle{DataIO_i_fade} = [DataIO_i,draw=black!30,fill=white,text opacity=0.3]

% Edges
\tikzstyle{DataLine} = [color=black!40,line width=5pt]
\tikzstyle{ProcessHV} = [-,line width=1pt,to path={-| (\tikztotarget)}]
\tikzstyle{ProcessTip} = [-,line width=1pt]
\tikzstyle{ProcessHVA} = [->,line width=1pt,to path={-| (\tikztotarget)}]
\tikzstyle{ProcessHVARev} = [<-,line width=1pt,to path={-| (\tikztotarget)}]
\tikzstyle{ProcessTipA} = [->,line width=1pt]

% **OLD** Faded edges
%\tikzstyle{DataLine_fade} = [DataLine,color=black!10]
%\tikzstyle{ProcessHV_fade} = [ProcessHV,color=black!30]
%\tikzstyle{ProcessTip_fade} = [ProcessTip,color=black!30]

% Matrix options
\tikzstyle{MatrixSetup} = [row sep=3mm, column sep=2mm]

% Declare a background layer for showing node connections
\pgfdeclarelayer{data}
\pgfdeclarelayer{process}
\pgfsetlayers{data,process,main}

% A new command to split the component text over multiple lines
\newcommand{\MultilineComponent}[3]
{
	\begin{minipage}{#1}
	\begin{center}
		#2 \linebreak #3
	\end{center}
	\end{minipage}
}

\newcommand{\FixedWidthText}[2]
{
	\begin{minipage}{#1}
	\begin{center}
		#2
	\end{center}
	\end{minipage}
}

% A new command to split the component text over multiple columns
\newcommand{\MultiColumnComponent}[5]
{
	\begin{minipage}{#1}
	\begin{center}
	#2 \linebreak #3
	\end{center}
	\begin{minipage}{0.49\textwidth}
	\begin{center}
	#4
	\end{center}
	\end{minipage}
	\begin{minipage}{0.49\textwidth}
	\begin{center}
	#5
	\end{center}
	\end{minipage}
	\end{minipage}
}


\begin{tikzpicture}

	% Use a matrix to line up all the nodes
	\matrix[MatrixSetup]{
		% First row
		& 
		\node [DataIO] (Input) {2: Baseline design}; 
		& 
		&
		&
		& 
		& 
		&
		\\
		% First row
		\node [Function] (pyHyp) {1: Pre-processing}; 
		& 
		&
		\node [DataIO] (pyHyp-pyGeo) {3: Free-form deformation points};  
		&
		&
		\node [DataIO] (pyHyp-IDWarp) {4: Volume mesh}; 
		& 
		& 
		&
		\\
		% Second row
		\node [DataIO] (Output) {8: Optimized design}; 
		& 
		\node [Optimization] (Opt) {\MultilineComponent{2.0cm}{2, 7$\rightarrow$3:}{Optimizer}}; 
		&
		\node [DataIO] (Opt-pyGeo) {3: Updated design variables}; 
		&
		&
		& 
		&
		&
		\\ 
		% Third row
		& 
		\node [DataIO] (pyGeo-Opt) {\FixedWidthText{2.5cm}{7: Geometric constraints \& their derivatives}}; 
		&  
		\node [Function] (pyGeo) {3: Geometry parameterization}; 
		&
		&
		\node [DataIO] (pyGeo-IDWarp) {4: Updated design surface}; 
		& 
		&
		&
		\\
		% Fourth row
		& 
		& 
		& 
		&
		\node [Function] (IDWarp) {4: Volume mesh deformation}; 
		& 
		&
		\node [DataIO] (IDWarp-flowSolver) {5: Updated mesh}; 
		&

		\\ 
		% Fifth row
		& 
		\node [DataIO] (flowSolver-Opt) {\FixedWidthText{2.5cm}{7: Values of objectives \& constraints}}; 
		& 
		& 
		&
		& 
		& 
		\node [Function] (flowSolver) {5: Flow simulation}; 
		&
		\node [DataIO] (flowSolver-adjointSolver) {6: State variables}; 
		\\
		% Sixth row
		& 
		\node [DataIO] (adjointSolver-Opt) {\FixedWidthText{2.5cm}{7: Derivatives of objectives \& constraints}}; 
		& 
		&
		& 
		& 
		& 
		&
		\node [Function] (adjointSolver) {6: Adjoint computation}; 
		\\
	};
	
	% Create a chain to outline process
	{ [start chain=process]
		\begin{pgfonlayer}{process}
		\chainin (pyHyp);
		\chainin (Opt)		  [join=by ProcessHVA];
		\chainin (Opt-pyGeo)          [join=by ProcessTipA];
		\chainin (pyGeo)          [join=by ProcessTipA];
		\chainin (pyGeo-IDWarp)          [join=by ProcessTipA];
		\chainin (IDWarp)  [join=by ProcessTipA];
		\chainin (IDWarp-flowSolver)  [join=by ProcessTipA];
		\chainin (flowSolver)     [join=by ProcessTipA];
		\chainin (flowSolver-adjointSolver)     [join=by ProcessTipA];
		\chainin (adjointSolver)  [join=by ProcessTipA];
		\chainin (adjointSolver-Opt)  [join=by ProcessTipA];
		\chainin (flowSolver-Opt)  [join=by ProcessTipA];
		\chainin (pyGeo-Opt)  [join=by ProcessTipA];
		\chainin (Opt)		  [join=by ProcessTipA];
		\chainin (Output)	  [join=by ProcessTipA];
		\end{pgfonlayer}
	}
	
	\begin{pgfonlayer}{data}
		% Vertical Edges
		\path 
		(Input)        edge [DataLine] (adjointSolver-Opt)
		(pyHyp-pyGeo)    edge [DataLine] (pyGeo)
		(pyHyp-IDWarp) edge [DataLine] (IDWarp)
		(IDWarp-flowSolver) edge [DataLine] (flowSolver)
		(flowSolver-adjointSolver) edge [DataLine] (adjointSolver)
		%% Horizontal Edges
		(pyHyp)        edge [DataLine] (pyHyp-IDWarp)
		(Output)       edge [DataLine] (Opt-pyGeo)
		(pyGeo-Opt)        edge [DataLine] (pyGeo-IDWarp)
		(IDWarp)        edge [DataLine] (IDWarp-flowSolver)
        (flowSolver-Opt)        edge [DataLine] (flowSolver-adjointSolver)
		(adjointSolver)        edge [DataLine] (adjointSolver-Opt);
	\end{pgfonlayer}
		
\end{tikzpicture}

\end{document}